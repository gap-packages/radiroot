%%%%%%%%%%%%%%%%%%%%%%%%%%%%%%%%%%%%%%%%%%%%%%%%%%%%%%%%%%%%%%%%%%%%%%%%%
%%
%W  install.tex          Radiroot documentation           Andreas Distler
%%
%H  $Id:$
%%
%Y  2005
%%

%%%%%%%%%%%%%%%%%%%%%%%%%%%%%%%%%%%%%%%%%%%%%%%%%%%%%%%%%%%%%%%%%%%%%%%%%
\Chapter{Installation}

%%%%%%%%%%%%%%%%%%%%%%%%%%%%%%%%%%%%%%%%%%%%%%%%%%%%%%%%%%%%%%%%%%%%%%%%%
\Section{Getting and Installing this Package}

This package is available at

\begintt
http://www.icm.tu-bs.de/ag_algebra/software/distler/radiroot
\endtt

in form of a gzipped tar-archive. For the installation instructions see
Chapter~"ref:Installing a GAP Package" in the {\GAP} Reference Manual. 
Normally you will unpack the archive in the 'pkg' directory of your
{\GAP}-Version by typing:

\beginexample
    bash> tar xfz radiroot.tar.gz        # for the gzipped tar-archive
\endexample

%%%%%%%%%%%%%%%%%%%%%%%%%%%%%%%%%%%%%%%%%%%%%%%%%%%%%%%%%%%%%%%%%%%%%%%%%
\Section{Loading the Package}

To use the {\Radiroot} Package you have to request it explicitly. This  is
done by calling

\beginexample
gap> LoadPackage("radiroot");
-----------------------------------------------------------------------------
Loading  RadiRoot 1.0 (Roots of a Polynomial as Radicals)
by Andreas Distler (a.distler@tu-bs.de).
-----------------------------------------------------------------------------
true
\endexample

The `LoadPackage' command is described  in  Section~"ref:LoadPackage"  in
the {\GAP} Reference Manual.

If you want to load the {\Radiroot} package by default, you  can  put  the
`LoadPackage' command  into  your  `.gaprc'  file  (see  Section~"ref:The
.gaprc file" in the {\GAP} Reference Manual).

%%%%%%%%%%%%%%%%%%%%%%%%%%%%%%%%%%%%%%%%%%%%%%%%%%%%%%%%%%%%%%%%%%%%%%%%%
%%
%E
